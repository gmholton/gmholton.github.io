%%%%%%%%%%%%%%%%%%%%%%%%%%%%%%%%%%%%%%%%%%%%%%%%%%%%%%%%%%%%%%%%%%%
% Template for Linguistics Interdisciplinary Studies Proposal
% created 2021-03-24 by holton@hawaii.edu
%
\documentclass[11pt,oneside]{article}
\usepackage[utf8]{inputenc}
\RequirePackage[letterpaper,margin=1in]{geometry}
\RequirePackage[]{makecell}

\begin{document}
    { \centering\LARGE \textbf{\uppercase{Interdisciplinary Studies Proposal}}\par}\vspace{2em}
    \begin{table}[h]
        \begin{tabular}[t]{p{4in}l}
            \textbf{Proposed B.A. in Interdisciplinary Studies} \hspace{1cm} 
             & \makecell[l]{
%%%%%%%%%%%%%%%%%%%%%%%%%%%%%%%%%%%%%%%%%%%%%%%%%%%%%%%%%%%%%%%%%%%
%           student contact information 
             Graduation Date: Spring 2023 \\ \\   % anticipated graduation date (estimate)
             Student name \\  
             Student address \\ Honolulu, HI   96822 \\ 
             808-555-5555\\             % best contact telephone number
             UH ID: XXXX-XXXX \\        % student ID
             student@hawaii.edu } \\   % UH email address
%%%%%%%%%%%%%%%%%%%%%%%%%%%%%%%%%%%%%%%%%%%%%%%%%%%%%%%%%%%%%%%%%%%
        \end{tabular}
    \end{table}

\subsection*{Title: Linguistics}

\setlength{\parindent}{0em}\setlength{\parskip}{0.5em}

% [short explanation of how you developed your interest in this field] 
\indent 
I have always had an interest in language…. During a general linguistics class I took this past semester, I figured out that the subfields of … most pique my interest. In addition, after studying \_ language for \_ years, I feel I need to increase my fluency in order to better understand the historical processes in the evolution of different languages. I would like to create a broad interdisciplinary program with a focus on linguistics.

% [description of your major in a cohesive, but concise, form] 
For this purpose, I wish to have a broad background in phonetics, phonology, morphology, syntax and semantics that are the core of subfields of language research. I would also benefit from a strong foundation in general linguistic theory and a basic training in articulatory phonetics, or how sounds are produced; morphology, or word formation; and syntax, or how syntactic units convey ideas and meaning. Moreover, a study of psycholinguistics would be beneficial because it will familiarize me with the mental processes involved in language production and acquisition. Since I wish to focus on the \_ language, I would benefit from improving my fluency in both the spoken and written language.  Equally important would be to learn more about language documentation and conservation and how to contribute toward sustaining Indigenous languages. Finally, it would be useful to learn about the theoretical concepts as they relate to learning and teaching of second languages because I can apply that knowledge to language reclamation. A basic training in language analysis for teaching second languages would also be useful in my linguistic studies.
	
% [future plans] 
After graduation, I may pursue an advanced degree in linguistics or pursue work opportunities in the field of linguistics within the United States.




\newpage
\section*{Agreement with Interdisciplinary Studies Program}
\subsection*{Major Equivalent}

\begin{table}[h]
\begin{tabular}{llr}
% these five courses are required
LING 320 &	General linguistics	& 3\\
LING 410 &	Articulatory phonetics	& 3\\
LING 420 &	Morphology	& 3\\
LING 421 &	Introduction to Phonological Analysis	& 3\\
LING 422 &	Introduction to Grammatical Analysis	& 3\\
% choose two additional LING courses from the list at http://ling.hawaii.edu/application-procedure/
% LING 331 Computer Applications (3)
% LING 344 Languages of the World (3)
% LING 346 The Philippine Language Family (3)
% LING 347 Pidgin and Creole Languages (3)
% LING 412 Psycholinguistics (3)
% LING 415 Language and Gender (3)
% LING 416 Language as a Public Concern (3)
% LING 423 Cognitive Linguistics (3)
% LING 430 Animal Communication (3)
% LING 431 Computational Modeling (3)
% LING 441 Meaning (3)
% LING 445 Polynesian Language Family (3)
% LING 451 Induction of Linguistic Structure (3)
% LING 470 Children’s Speech (3)

LING XXX & \dots &	3\\
LING XXX & \dots &	3\\
		

% 15 credit hours of course work in two other departments suited to students’ areas of interest and post-graduation plans: e.g., anthropology, English, foreign languages, information and computer sciences, music, philosophy, psychology, second language studies, etc.
	
& & \\
& & \\
& & \\
& & \\
& & \\
	
\hline		
&	\hfill total credits	& $\geq$ 36\\

\end{tabular}
\end{table}

\subsection*{Unfinished Prerequisites or Special Requirements to satisfy the IS Degree}
% For example, if you have not yet taken LING 102 (a prerequisite for LING 320), list that course here.

\vfill
In signing this agreement, I acknowledge that it is my responsibility to meet with my Interdisciplinary Studies advisor regarding general education and program requirements for graduation.  This includes a preliminary advising session and the Graduation Check. I also acknowledge that there is no double-dipping in double majors, minors, or certificates.\\[1em]


\begin{table}[h]
    \centering
    \begin{tabular*}{\textwidth}{p{12cm}p{3cm}}
        \hline
         Student Name & Date  \\[2em]
         \hline
         Gary Holton, Department of Linguistics & Date \\[2em]
         \hline
        Amy Schiffner or Jaishree Odin, Interdisciplinary Studies  & Date \\[1em]
    \end{tabular*}
\end{table}


\vspace{1cm}
*Mark approved transfer courses (if any) with an asterisk and indicate where they were taken in this footnote. 


\end{document}
